\documentclass[12pt]{article}
\usepackage[english]{babel}
\usepackage[utf8]{inputenc}
\usepackage[babel]{csquotes}
\usepackage{graphicx}
\usepackage{etoolbox}
\usepackage{changepage}
\usepackage{titlesec}
\usepackage{hyperref}
\usepackage[parfill]{parskip}
\usepackage[margin=1in]{geometry}
\usepackage{times}
%\usepackage{xunicode}
\usepackage[dvipsnames,table]{xcolor}
\usepackage{tikz}
\usetikzlibrary{positioning,arrows,shapes,fit}
\usepackage{standalone}
\usepackage{textcomp}

% sorting=none will order citations in order of appearance
% use sorting=nyt to sort by name, year, title
\usepackage[backend=biber,autolang=hyphen, bibencoding=inputenc, isbn=false, uniquename=false, sorting=none, sortcites=true, style=numeric, autocite=superscript, url=false]{biblatex} % biblatex setup
\addbibresource{computing.bib}
\appto{\bibfont}{\footnotesize} % for 10pt bibliography

\titleformat*{\section}{\normalsize\bfseries} % Makes section titles 12 pt font
\titleformat{\section}{\normalsize\bfseries}{}{}{}[]
\titleformat{\subsection}[runin]{\bfseries}{}{}{}[.]
\titleformat{\subsubsection}[runin]{\itshape}{}{}{}[.]

\let\textcite=\autocite

\title{\large \textbf{Leveraging Historical Ties between Cognitive Science and
  Computer Science to Guide Programming Education}}

% Author info isn't included for the Annual Conference but some regional conferences might request it.
\author{}
%\author{\normalsize Author Name\\
%\normalsize email@example.com\\
%\normalsize Name of Your Department\\\
%\normalsize Your Institution Name}
\date{}

\makeatletter % This gets the margins for the title set.
\patchcmd{\@maketitle}{\begin{center}}{\begin{adjustwidth}{0.5in}{0.5in}\begin{center}}{}{}
\patchcmd{\@maketitle}{\end{center}}{\end{center}\end{adjustwidth}}{}{}
\makeatother

%% TODO: be more explicit about learning framework

%% TODO: bring in formative assessment

\begin{document}
\raggedright
\maketitle
\thispagestyle{empty}
\pagestyle{empty}

% conflict between teaching programming syntax, teaching ``how to program'', teaching high-level concepts like abstractions and how to ``do'' problem solving.
\section*{Coding for everyone}
In the past few years, there has been increasing interest in
encouraging more people, regardless of background, to learn to
program. In fact, President Obama recently made a statement calling on
all children to have the opportunity to learn about computer
science\autocite{whitehouse_computer_2016}. Sites such as code.org
promote CS education opportunities for all, citing statistics about
STEM jobs and arguing that in the 21st century, knowledge about
computer science is foundational\autocite{code.org_every_2016}. Many
institutions of higher education either have programming requirements
for non-computer-science majors, or have been expanding programming to
non-majors\autocite{rich_cs1_2004,forte_motivation_2005,guzdial_design_2005}. At
the authors of this paper's institution, all first year general
engineering students are required to complete an introductory
programming module as part of their first-year engineering course, and
in general ``basic programming skills'' is a common desired outcome
across engineering curricula\autocite{reid_classification_2014}.

This coding-for-all paradigm juxtaposed with the observation that
technology is changing rapidly, and the half-life of technical skills
is decreasing\autocite{nae_educating_2004}, might seem to indicate that
teaching programming is a questionable use of class time: the language
and paradigm students are taught will likely not be the same they are
asked to use in their first job post-graduation. However, if we
acknowledge that learning to program is more than simply learning to
write code, and in particular, if we draw on work from the early era of
computing which closely linked programming to problem solving and
cognitive skills, we can argue that regardless of language or
paradigm, learning to program may actually be a way to learn to
problem solve, organize knowledge, and reason about processes.

At the dawn of the computing revolution, visionaries predicted that
computers had the potential to ``augment the human intellect''
\autocite{engelbart_augmenting_1962}, a much more expansive view than
simply becoming tools to automate tedious calculations. In other words, computing allows for new ways of dealing with
complexity, and provides new metaphors for thinking about
systems\autocite{kay_history_1996}. Early work demonstrated that with
the aid of computers, children could engage with tasks and concepts
that had previously been considered ``advanced'', and through
interacting with computers, children could become more aware of their
own cognitive processes \autocite{papert_mindstorms_1980}. Thus, the
fact that specific technologies may be obsolete by the time students
graduate may be beside the point, there are lasting cognitive benefits
that learning to interact with a computer can provide.

While there are some recent studies which investigate programming
education as a means of teaching problem-solving, there is also a
large body of research that primarily focuses on teaching students the
details of how to code. We argue that in most cases these details are
less important than centering programming education around
problem-solving skills. If we make learning problem-solving skills a
goal of introductory programming courses, then all aspects of the
course, from the course objectives, to the curriculum used, to the
assessments administered, must be chosen intentionally with this goal
in mind. 

In this paper we will present a brief summary of current published
work on introductory programming education. In particular, we will
explore the alignment of course objectives, pedagogical strategies,
and assessments, and will offer a potential framework to help think
about both the design and evaluation of introductory programming
courses. We will then provide a hypothetical case study of a
introductory programming course organized around this framework and
discuss implications of different instructional strategies through the
lens of a theory of cognitive processes called cognitive load theory.

%“Computing enters as a new representation system with new and useful
%metaphors for dealing with complexity, especially of systems”
%\autocite{kay_history_1996}

% “I wasn't smarter but I had a much better internal thinking tool to
% amplify my abilities.” \autocite[][p.526]{kay_history_1996} about his
% ability to quickly solve a problem using LISP.

% “the 'hacker phenomenon'” \autocite[][p.544-5]{kay_history_1996}

% Difficulty novices experience have “little to do with features of the
% programming languages used, and everything to do with the difficulties
% novices have thinking the secial way that good programmers think”
% \autocite[][p.546]{kay_history_1996}

% “However, a program is not just for the computer: A program that
% follows the rules of discourse better enables a human reader to
% reconstruct the explanation that the program writer was following in
% initially developing the mechanism.”
% \autocite[][p.857]{soloway_learning_1986}

\section*{Methods}
Assuming that there would be more comprehensive research of
introductory programming in the computer science community, we began
with targeted searches of the ACM digital library for the key phrase
`introductory programming'. Once we found key papers that seemed to
indicate a call to action
\autocite{mccracken_multinational_2001,kolling_problem_1999},
or a survey of practices
\autocite{pears_survey_2007-1,robins_learning_2003} we used
the `cited by' list to find additional relevant papers.

Because the literature search of introductory programming resulted in
a variety of examples of how and why programming was taught, we needed a
documented framework to consistently evaluate each one and guide our
hypothetical course design. Starting from
\citeauthor{jamieson_creating_2009}'s
(\citeyear{jamieson_creating_2009}) call to ground educational
research activities in how people learn, we identified the concept of
\emph{constructive alignment} as a promising framework to evaluate and
design learning environments. Following the references in
\citeauthor{jamieson_creating_2009}'s report as well as conducting
targeted literature searches for ``constructive alignment'' we were
able to find descriptions of this
framework\autocite{jamieson_creating_2009,pellegrino_rethinking_2006,felder_designing_2003,biggs_enhancing_1996}.
%% TODO: mention that we chose the framework after analyzing the above
%% citations, i.e. it wasn't a direct pull from jamieson

%% Despite hope in the 1990s that computing and programming had potential
%% for bringing more undrerepresented groups into STEM fields, it has
%% become one of the *least* diverse fields.

% \textcite{turkle_epistemological_1990} were optimistic that paradigm
% shift to object oriented programming that began in the 1990s would
% result in more epistemological diversity since at its core, OOP as a
% design philosophy is compatible with what they call a “bricolage”
% strategy towards design (e. g. the concept of message passing between
% agents invokes association with a dialog or negotiation which is how
% people following a bricolage style see themselves interacting with the
% machine). Unfortunately, based on more current literature regarding
% how OOP is taught the elements of OOP that most align with a bricolage
% style are left out while concepts of OOP that align with the
% traditional “planner” style: encapsulation, inheritance are
% emphasized.

\section*{Current state of research}
The bulk of literature about teaching introductory programming (CS1)
revolves around what paradigms are ``best'' (e.g. procedural
vs. object oriented
\autocite{bailie_objects_2003,moritz_objectsfirst_2005,reges_back_2006}),
which language is ``best'', often times with regard to a particular
paradigm (e.g. OOP with Python
vs. Java\autocite{goldwasser_teaching_2008,koulouri_teaching_2014}),
and best teaching practices (e.g. use of ``pair programming''
environments\autocite{williams_support_2001,mcdowell_effects_2002,thomas_code_2003,chong_social_2007,salleh_empirical_2011}). However,
the \emph{reasons} for picking one paradigm over another tend to be
about industry demand, rather than intentionally choosing a paradigm
to support the learning objectives of the course. This behavior,
adopting a popular practice into a learning environment without
critically analyzing whether or not is a good fit, is by no means
exclusive to programming education. For example, flipped
classrooms\autocite{bishop_flipped_2013}, in which students are asked
to watch recorded lectures at home and use class time for working on
activities, has recently been a hot topic in higher education and
there have been a continuous number of examples of well-intentioned
practitioners flipping their classroom without adjusting course
objectives, content, and activities to work under the
flipped-classroom model\autocite{desantis_students_2015}. Likewise,
anyone can teach elements of object oriented programming in their
introductory programming courses, but unless critical aspects of the
course are adjusted: background reading, rethinking what the
``fundamental concepts'' are, the types of problems assigned, and
methods of assessment, the incorporation of OOP will likely create
confusion for students and frustration for
instructors\autocite{kolling_problem_1999}.

%% TODO: more directed title
%% TODO: possibly use subheadings to ``tweet'' the major takeaways from each section,
%% this could then be connected at the end to bullet points of suggestions
\section*{Trends in programming education research}
A \citeyear{robins_learning_2003} study by
\citeauthor{robins_learning_2003} identifies four trends in research
of programming education: 1) distinguishing between novice and expert
programmers with emphasis on deficiencies of novices, 2) the
distinction between knowledge and strategies, 3) the distinction
between program comprehension and generation, and 4) a comparison of
OOP to procedural styles of
programming\autocite{robins_learning_2003}. They found that the
biggest challenge for novice programmers was not in understanding the
basic concepts, but knowing how to apply them. This finding is
consistent with current understanding of how we
learn\autocite{bransford_how_2000,ambrose_chapter_2010}, and with
studies of expert/novice behavior in other domains, such as
statics\autocite{litzinger_cognitive_2010}. In both cases the
suggestion is that helping students learn strategies that work is
crucial to helping them become \emph{effective} novices, i.e. devote
instructional time to problem-solving strategies and how to apply
them.

A \citeyear{mccracken_multinational_2001} study by
\citeauthor{mccracken_multinational_2001}\autocite{mccracken_multinational_2001},
which concluded that regardless of nation or institution, most
students do not know how to program by the end of their introductory
programming courses, sparked much conversation and a push to reform
introductory programming in the next decade. This phenomenon is by no
means exclusive to introductory programming: disconnect between what
we think we are teaching and what students learn has been observed
across
engineering\autocite{streveler_learning_2008,flynn_engineering_2014}. Possible
ways of addressing this disconnect include recognizing differences in
student's prior knowledge, helping students see connections between
abstract concepts, and relating difficult concepts to ones students
already know\autocite{streveler_learning_2008,slotta_helping_2006}.
%% TODO: connect mccracken to engineering ed about missconceptions
%% (CHEER chapter and Streveler paper)

While the \citeauthor{mccracken_multinational_2001} study validated the
experiences of many introductory programming instructors, it did not
result in critical change to address the problems. Rather, what we
have seen since the 1990s has been an increasing interest in where
object oriented programming should be placed within the curriculum
with debates between objects-first vs procedural-fist, but little
question of why object oriented design should be included at all other
than that it continues to be a popular paradigm in industry.

There remains a desire that introductory programming courses should
teach problem-solving skills, though only a few authors present
evidence that the implementation of this goal is informed by cognitive
science\autocite{kramarski_cognitivemetacognitive_1997,reed_incorporating_1998,muller_supporting_2008}. One
thread in particular has received regular attention:
\citeauthor{east_patternbased_1996} propose a pattern-based approach
that places emphasis on recognizing and learning to work with
algorithmic patterns over syntax and design
paradigms\autocite{east_patternbased_1996}. This approach closely
aligns with understanding of differences in expert/novice behavior by
helping students identify patterns of solutions and learning how and
when to apply them to particular
problems\autocite{east_patternbased_1996,muller_pattern_2005,wallingford_first_1996,reed_incorporating_1998}.

\citeauthor{pears_survey_2007-1} conclude their
\citeyear{pears_survey_2007-1} literature survey with the observation
that despite the volume of work exploring the implementation of
introductory programming courses, ``there is little systematic
evidence to support any particular
approach.''\autocite{pears_survey_2007-1} We think this could be due
to two reasons:

\begin{enumerate}
\item There has been a lack of consensus on what the goals of an
  introductory course should be: the ability to write a complete,
  working program, the ability to answer conceptual questions about
  computing and computer science, or the ability to analyze problems
  and design solutions (problem-solving). The ``appropriate'' approach
  to teaching would depend on the goal.

\item Even if goals were clearly stated, it is likely that many
  different approaches could be used successfully if chosen
  intentionally to align with the goals and measurements of success
  (assessment). In many of the studies, it is not immediately clear if
  the measures of success used are appropriate for the stated
  objectives. That is to say, the assessments used might not always be
  measuring all aspects of the desired learning objectives.
\end{enumerate}

Finally, in \citeyear{koulouri_teaching_2014},
\citeauthor{koulouri_teaching_2014} conducted a quantitative analysis
of different approaches to teaching introductory programming by
varying three aspects: the language used, providing formative
feedback, and introducing explicit problem solving
training\autocite{koulouri_teaching_2014}. They found that choosing a
language with a simpler syntax (Python instead of Java), and
introducing explicit problem-solving training, improved student
learning. They also found, however, that providing more formative
assessment did not seem to help. They speculate that the reason for
the latter finding could be that students do not spend time looking at
formative feedback, or do not understand the feedback that is given.

We wish to build upon the past reviews of the field by suggesting an
approach to designing learning systems that may help to consistently
address some of the shortcomings observed, and bring understanding to
the successes. A large number of papers surveyed contain discussions
of course objectives, pedagogical tools used, and assessments given,
but rarely are these three components discussed together. We know from
the engineering education literature that an effective learning system
depends on the alignment of these three
components\autocite{jamieson_creating_2009,pellegrino_rethinking_2006}.

\subsection*{Components of a learning
system}\label{sec:components-of-a-learning-system}
%% Reviewer: short
A popular way to visualize an effective learning system is as a triad
of objectives, instruction, and
assessment\autocite{felder_designing_2003}, shown in Figure
\ref{fig:learning-system}. The design of these three components is not
sequential: each informs the other.

\begin{figure}[!ht]
  \centering
  \includestandalone[width=\textwidth]{felder_framework}%
  \caption{Learning system as derived from
    \citeauthor{felder_designing_2003}\autocite{felder_designing_2003}}
  %%TODO: as derived from original figure
  \label{fig:learning-system}
\end{figure}

\subsection*{Effective learning systems}\label{sec:effective-learning-systems}

Assessment results can help inform changes to course objectives and
instructional methods\autocite{felder_designing_2003}. Similarly, as
instructional methods improve, the objectives may be modified to aim
for higher levels of understanding. Being intentional about these
interactions is crucial to an effective learning system; if the
assessment addresses lower levels of learning than the objectives
specify, ``the system will be driven by backwash from testing, not by
the curriculum.''\autocite[][p. 350]{biggs_enhancing_1996}. For
example, if an objective of a programming course is that learners will
develop problem solving skills, but assessment only consists of tests
of programming concepts and syntax, teaching/learning activities would
tend to center around those that promote understanding of these
concepts and exclude activities that might have a better chance at
promoting general problem solving\autocite{biggs_enhancing_1996}.

\section*{A hypothetical case study}
As noted, teaching OOP is a common area of interest in the
literature. While \citeauthor{robins_learning_2003} note there is a
lack of evidence supporting any claim that OOP makes modeling problems
any easier\autocite{robins_learning_2003}, we will incorporate it into
a hypothetical introductory programming course to explore how its
inclusion would affect the constructive alignment between course
objectives, instructional methods, and assessment.

It should be strongly noted that we are not advocating for OOP as a
``best'' teaching paradigm. Rather, we want to point out that from
what has been reported, teaching OOP effectively can be
challenging. What many implementations seem to miss is that OOP
defines an entirely different paradigm for problem solving than is
used in procedural programming\autocite{kolling_problem_1999}. As
should not be surprising, there are some problems for which OOP
provides a good set of tools to solve in an efficient, elegant manner,
namely those with solutions that naturally map to real-world
objects\autocite{raymond_art_2003}. Different problems may be better
solved using a procedural, functional, or data-flow
paradigm\autocite{kay_history_1996,felleisen_functional_2009,crestani_experience_2010}. Ultimately,
our goal should be to help students analyze problem structures and
make informed decisions regarding what paradigm would be best for a
particular problem. What is not clear from the literature is whether
or not this consideration is made when designing problems for a
particular course. Indeed, in some cases there is evidence that OOP
concepts are added to a course ``on top of'' older content, without
necessarily changing the problem
sets\autocite{kolling_problem_1999,vilner_fundamental_2007}.

\subsection*{Cognitive load theory}
Cognitive load theory (CLT) is based on the premise that the capacity
of our working memory, where all conscious cognitive processing takes
place, is finite and relatively small: we are able to manage 2-3
interacting elements at a time. Examples of ``interacting elements''
in the current context would be recalling and using the syntax and
grammar of a particular programming language, reasoning about the
behavior of a single object in an OOP environment, or reasoning about
a solution to a particular high-level problem. CLT posits that in
light of this finite capacity of working memory, for learning to
occur, all aspects of a task must fit into this finite space. While
some contributors to cognitive load are necessary for the task at
hand, others are not\autocite{paas_cognitive_2003}. Cognitive load can
be divided into three different categories: \emph{intrinsic} cognitive
load is due to the complexity of the task itself and generally can not
be reduced without simplifying the task. \emph{Extraneous} cognitive
load is associated with work that is due to the instructional method,
but not part of the task at hand. For instance, if the educational
goal is to complete a programming task, but learners must search and
interpret reference material in order to do so, the search and
interpretation of this material would be considered extraneous
cognitive load. Finally, \emph{germane} cognitive load is also a
result of the instructional design, but unlike extraneous load,
germane load enhances the learning process. For instance, if a
learning goal is that students are able to navigate and use reference
material, than the previous example would be considered germane
cognitive load. What this example highlights is that whether
complexity due to the instruction design is extraneous or germane
depends on the specific learning goals. The role of the instructional
designer then is to be aware of intrinsic load, reduce extraneous load
and intentionally managing germane load.

\subsection*{Object oriented programming}
%% Reviewer: more explicit connections to cogsci
Alan Kay originally conceived of object oriented programming as
\emph{one} way to think about complex
systems\autocite{kay_history_1996}. The underlying metaphore of Kay's
conception for OOP was the biological cell: each object is designed to
accomplish a specific set of tasks such that the complexity of these
tasks is encapsulated within the object. Objects interact with one
another by sending and receiving messages, and systems of these
objects are assembled to solve complex
problems\autocite{kay_history_1996}.

%%TODO: link to knowledge/skills interaction literature
%%TODO: bring in knowledge organization, focus on concepts vs. procedures

In several reports of attempts to incorporate OOP into introductory
programming courses, practitioners worried that time devoted to OOP
concepts would necessarily reduce time spent on the ``basics of
programming'', usually defined as conditional statements, loops, and
in some cases pointers. This concern is misplaced; under an OOP
paradigm the ``basics of programming'' are not the same as when
working from a procedural paradigm. OOP defines a completely different
set of tools for problem solving, and it is these tools and concepts
that become the ``basics'': objects, messages, inheritance, and
polymorphism\autocite{kolling_problem_1999,kay_history_1996}.

For instance, if taking a patterns-based approach to teaching
programming, one could use either a procedural paradigm or an OOP
paradigm, but the fundamental patterns would be quite different
depending on the paradigm\autocite{wallingford_first_1996}. Useful
procedural patterns might be:

\begin{description}
\item[read-evaluate-print] reading a piece of data, evaluating it, and output the result
\item[guarded-action] if a guard-condition is satisfied, take action.
\end{description}

While useful OOP patterns might be:

\begin{description}
\item[state]  provide controlled access to a body of data
\item[view] decouple an object's state from its representation
\item[decorator] add functionally to an object without modifying its internal structure.
\end{description}

Ultimately, if our goal is to teach problem solving in general, we
would be doing a grave disservice to our students to leave them with
the impression that there is only one paradigm, whether it be OOP,
procedural, functional, or something not yet invented. As previously
mentioned, OOP and procedural paradigms are comprised of fundamentally
different concepts for problem solving, and so again is the functional
paradigm. Unsurprisingly, there is no paradigm that is the ``best''
fit for every problem. Leaving students with the belief that only one
paradigm exists makes it very difficult for them to learn how to
intentionally select and apply the ``best'' paradigm to solve a
particular problem. \citeauthor{kolling_problem_1999} argues that
performing a paradigm shift from procedural style to OOP is difficult,
and so concludes that if OOP is a desired outcome then it should be
taught \emph{first} since the reverse shift, from OOP to procedural,
is conceptually easier. What this recommendation neglects to account
for however is that both paradigms depend upon authoring programs in
some language, and this authoring process takes a certain amount of
cognitive capacity. To justify an objects-first approach we would have
to verify that the cognitive load associated with OOP concepts
themselves combined with the cognitive load of learning to author a
program in a particular environment, does not exceed the learner's
working memory capacity.

In a superficial way, CLT theory would suggest that OOP is a helpful
paradigm for managing cognitive complexity: objects encapsulate
complexity by hiding the details of implementation behind a simple
interface. For example, an object that represents a list of items may
have operations to add and remove items from the list. Conceptually we
can use this object with our common understanding of what it means to
``add'' an ``item'' to a ``list''. Critically, we do \emph{not} need
to be aware of how the adding and removal of the item is actually
implemented in code, or even how the storage is structured in memory
(e.g. an array or a linked-list). We refer to this use of CLT to
justify the reduction of complexity of OOP as superficial because it
focuses on only the conceptual paradigm itself without regard to
implementing it in some programming environment. In practice, we
usually ask our students to implement OOP concepts in a particular
programming language. That task requires not only understanding OOP as
a paradigm, but also working with the programming environment,
authoring a program, understanding the syntax and grammar of the
language, and most likely debugging it as well.

\subsection*{Cognitive challenges teaching OOP}
A behavior that characterizes expert programmers is the ability to
easily shift between different abstraction layers of a problem
space\autocite{armoni_teaching_2013}. This ability may be especially
crucial to becoming fluent in OOP as the concepts are somewhat further
removed from the actual syntax than in other paradigms. For example,
in the procedural paradigm the central concept is that of the
procedure, or subroutine. Subroutines are simply collections of
imperative commands grouped together under a label. Thus, if someone
can write and understand imperative code, the conceptual leap to
understanding the procedural paradigm is relatively small. In
contrast, to understand objects in OOP one must understand the concept
of state and methods operating on the state. Both state and method are
implemented with code (generally imperative), but to understand how an
object functions one has to imagine its initial state, and then how
each method call would change this state. Cognitive load theory would
suggest that understanding the dynamic interaction of state and method
calls is more challenging than grouping lines of imperative codes into
re-usable procedures. Thus to be fluent in OOP design, a programmer
\emph{must} be able to switch between thinking at the object layer and
the program layer. Teaching abstraction is best done by doing it by
example, while being explicit about the abstractions
used\autocite{armoni_teaching_2013}. However, we must keep in mind the
limited capacity of working memory: if the cognitive load associated
with writing code and thinking about state and methods fills up a
learner's working memory, the learner may have difficulty grasping
concepts of abstraction, even when the instructor explicitly points
them out. To reduce the cognitive load associated with writing code
and working with object state and methods, these tasks must be
practiced until somewhat automated: to a point at which they can be
done without conscious thought\autocite{ambrose_chapter_2010}.

\subsection*{Language choice}
Choice of language does matter: If the learning objectives of an
introductory programming course are to learn problem solving concepts
then a language with a simpler syntax is preferable to one with more
complex syntax features\autocite{koulouri_teaching_2014}. The
explanation for this can be found in CLT. Given the current example of
a goal to teach problem-solving techniques, there will be cognitive
load associated with thinking about different strategies one could
use. Learning the syntax rules of a programming language adds to the
cognitive load, but it does not aid in the process of learning
problem-solving techniques. In other words, learning the syntax of a
programming language is an extraneous cognitive load in this
context. Thus, it is in the educator's best interest to minimize
cognitive load associated with features that do not directly help with
the learning objectives of the course and select a teaching language
with easy-to-learn syntax.

In an attempt to address the concern that the two primary languages,
C++ and Java, used to teach OOP in introductory courses may be too
complex for novice students, \citeauthor{goldwasser_teaching_2008}
select Python as a good choice for introductory programming claiming
that its relatively simple syntax allow new students to better engage
with the OOP concepts rather than getting bogged down by details of
the language\autocite{goldwasser_teaching_2008}.

\subsubsection*{Iconic versus textual languages}

In theory, an iconic programming environment such as
Scratch\autocite{resnick_scratch_2009} or
Alice\autocite{cooper_using_2003} should aid in the learning of higher
level concepts, such as those associated with OOP, since graphical
languages eliminates the extraneous cognitive load of dealing with
syntax and grammar of a language. In fact, there has been success in
using iconic languages to teach first year programming
\autocite{calloni_iconic_1997}, and research indicates that teaching
graphical programming first does not inhibit a student from later
transferring learned high level concepts to textual programming
languages\autocite{hundhausen_can_2009}.

\citeauthor{snook_incorporation_2005} describe efforts to use the Alice
programming environment to teach introductory programming
concepts\autocite{snook_incorporation_2005}. Interestingly a later
report assessing the efficacy of the curriculum using Alice indicated
that while pre-post testing indicated learning gains with the
environment, focus group data indicated a dissatisfaction with Alice
as an introductory language, resulting in a switch to
LabVIEW\autocite{lohani_reformulating_2011}. A possible explanation
given for the dissatisfaction of Alice was that students did not
perceive it as being a ``real'' programming environment they might use
in industry, while LabVIEW was. Certainly, helping students connect
the utility of skills they learn in the classroom to those that will
be important in their later career is important. In some cases
switching to a tool that is known to be used in industry may be
useful, though this approach seems shortsighted: it is unlikely that
all graduates will end up in an industry that uses a particular tool
(e.g. LabVIEW), so if that is the motivation for a particular tool we
must still address the utility of this tool to students who may never
have reason to use it again. A different strategy than swathing tools
is to better articulate \emph{why} a particular tool was chosen for
the learning environment. Students might have less resistance to an
educational environment such as Alice if they know that it can help
them learn high level concepts that \emph{will} be directly applicable
to their future career \emph{and} that the use of Alice as an
introductory environment will not hinder them from later transferring
those concepts to other environments.

\subsection*{Problem solving}
Developing problem solving skills was widely regarded as a common objective
for introductory programming courses. Achieving this objective is more
likely if the course begins with a focus on problem solving
strategies, before any programming is
introduced\autocite{koulouri_teaching_2014}. Separating
problem-solving instruction in this way has a number of benefits
grounded in cognitive science research: first, it serves to activate
prior knowledge\autocite{ambrose_how_2010}. With problem-solving
strategies fresh in their memory, learners will more easily apply them
when they begin programming. Second, it helps manage cognitive
load. Where as in the previous example of language choice, we reduced
cognitive load by choosing a language with a simpler syntax, here we
separate two tasks that we know individually demand a high level of
cognitive load for novices: learning problem-solving strategies, and
learning programming techniques.

% \section{Choice of problems to assign}
% %% Reviewer: short
% As previously noted, OOP provides a set of tools that naturally lend
% themselves to solving certain types of problems but not others. An
% educator teaching OOP should carefully select problems and challenges
% that make sense to be solved with an OOP approach. Of course,
% providing illustrative examples of problems that are \emph{not} best
% solved with an OOP approach would be useful provided there is also
% time for discussion of \emph{why} OOP might not be suited for certain
% problems and brief mention of alternative approaches.

\section*{Assessment}
%% Reviewer: short
Often times assessment becomes the weak point of a learning
system. Designing good assessment is
hard\autocite{lister_blooming_2000,traynor_synthesis_2005}, and often
good assessment can require significant time on the part of the
instructor. Because of this, assessment is often simplified, lowering
the effectiveness of the entire system.

If our course objective is to teach learners to solve problems with
programming, then an ideal assessment would be to give students a
problem to solve, ask them to write a program to solve it as well as
reflect upon the patterns they used in their solution. Asking for
reflection prompts students to engage in meta-cognition, becoming
aware of their own thinking process, which helps them become more
effective
learners\autocite{ambrose_how_2010,chi_selfexplanations_1989}. Of
course evaluating this type of assessment is much more time consuming
than a multiple-choice exam that can be automatically scored, or even
than evaluating that the finished program runs correctly, which can be
done automatically as well\autocite{cheang_automated_2003}.

\section*{Considerations and limitations}
In the above hypothetical example we had the luxury of complete
freedom in designing all aspects of an introductory programming
curriculum. We recognize this is almost always never the case in
practice; however, we argue that following a framework, such as
constructive alignment, as close as possible will help. Thinking about
the framework will also encourage critical engagement with the
constraints imposed by the institution. For example, there has been
much interest in assessment tools used for introductory programming
courses\autocite{lister_blooming_2000,lister_objectives_2001,sheard_exploring_2011},
and in particular, automated
assessment\autocite{cheang_automated_2003,traynor_synthesis_2005,ihantola_review_2010}. In
many cases the push towards automated assessment techniques is a
response to increasing class sizes that instructors typically have
little control over. In the event that a large class size dictates
that some form of automated assessment be used, following the
constructive alignment framework would constrain the types of course
objectives and instruction that would be feasible. For instance, if
the primary course objective was to teach conceptual understanding of
computer science, then a well-designed multiple-choice assessment
could be used
effectively\autocite{lister_blooming_2000,lister_objectives_2001}. Using
the framework of constructive alignment to critically evaluate the
alignment of assessment, objectives, and instruction, may help make
explicit some of the compromises that must be made for larger class
sizes. This positions course designers to be more intentional about
what compromises to learning are made.

\section*{Discussion and future work}

This literature review was by no means comprehensive, but we hope that
it touched upon some of the notable areas in which educators have been
experiencing challenges and successes in teaching introductory
programming. We have pointed out a number of possible connections
between computer science and cognitive science that may be worth
deeper investigation. While we have seen how cognitive load theory can
help reason about course design, there are also potentially
interesting connects between cognitive load theory and object oriented
design: objects are designed to encapsulate complexity to better
manage it. Additionally, further exploring concepts of knowledge
organization and differences between novices and exports may be a
fruitful area to investigate the use of teaching programming to aid in
analogical reasoning and abstract
thinking\autocite{bennedssen_abstraction_2008}.

An area that has not been addressed in this paper but has strong
connections is that of attention in
cognition\autocite{simon_role_1986}. There are strong parallels
between our limitations to focus our conscious attention at one or two
tasks at a time, and the serial processing inherent in most computing
and language design. Of particular interest is the move in hardware to
increasingly parallel designs to keep pace with Moore's law. Our
cognitive constraints of attention may suggest an inherent difficulty
in thinking and reasoning about parallel processing that must be
addressed if we want to prepare students for an environment that
demands parallel-processing solutions to programming problems. These
demands may shift focus to languages such as Erlang that are designed
for high levels of concurrency, result in more interest in functional
paradigms over OOP, as well as demand we adopt different metaphors for
computing altogether\autocite{stein_challenging_1999}.

\section*{Conclusion}

Recognizing that introductory programming is a common component of an
engineering curriculum, and that there is a national push to introduce
it to an even larger population, it is important to think carefully
about how we design programming education to impart the cognitive
skills that will generally benefit learners even after the specific
technologies they learn are no longer relevant. We strongly advocate
for the use of a framework such as constructive alignment, described
in this paper, to inform the choices for course objectives,
instructional techniques, and assessments. Using such a framework well
help identify components of a course that may be contributing
unnecessarily to cognitive load, or reasons why student outcomes might
not match expectations assumed from course objectives.

Analyzing a proposed curriculum through the lens of cognitive load
theory can aid in instructional decisions that will increase the
chance of achieving learning goals. Educators must critically review
the goals of their introductory programming courses: if goals are to
understand high level concepts such as OOP, but not necessarily
implement these concepts in textual languages, it may be prudent to
consider iconic languages for these courses. If however, goals include
learning OOP concepts \emph{and} implementing these concepts in
textual language, educators must be aware of the increased cognitive
load associated with each of these tasks and provide scaffolding to
help students automate aspects of each task individually before
expecting them to combine both in successful programs.

Taken together, using an established framework, such as constructive
alignment, to design and analyze a curriculum, and utilizing cognitive
load theory to help reduce extraneous cognitive load while being
intentional about inclusion of germane cognitive load, should help
instructional designers create learning environments in which students
are able to achieve learning goals. In some cases, intended goals may
have to be adjusted to account for the finite limitations of working
memory capacity: it is likely not feasible in the span of a single
semester to expect students with no programming experience to learn
the syntax and grammar of a new language \emph{and} higher level
concepts associated with a paradigm like OOP, \emph{and} be able to
switch between abstraction layers in the fluid manner necessary to
solving complex problems with programs. What this suggests is that
closer coordination is needed across courses in a curriculum so that
goals of one become background knowledge of subsequent courses,
resulting in successful achievement of program learning goals by the
time of graduation.

% \section{What are “the basics”}
% Several papers discuss the relative merits of a more “modern” paradigm
% such as OOP and procedural paradigms and what they refer to as “basic
% concepts” in programming. The “basics” tend to be conditional
% structures, looping structures, and arrays. However it is never noted
% that these “concepts” are only “basic” in the context of a particular
% paradigm, in particular procedural. For example, if we decided it was
% best to teach a functional paradigm in CS1 we would be hard pressed to
% defend conditional and looping structures as “basic”, rather we would
% consider recursion, evaluation, and higher-order-functions as “basic”
% concepts.

% \section{The relative dearth of functional-first approaches}
% Given the apparent focus on formalist design practices, what
% \textcite{turkle_epistemological_1990} refer to as “the planner”
% approach, it is interesting to note the relative lack of interest in
% using functional-first strategies. The functional programming paradigm
% is highly compatible with formal and abstract design methods and is
% closely associated with mathematical problem solving
% \autocite{backus_can_1978}. The relative lower interest in functional
% programming suggests that programming paradigms are chosen more for
% practical market concerns (e.g. OOP is popular in industry, adoption
% of functional styles have been slow to adoption largely due to early
% problems with speed), rather than any conscious link between design
% strategies, objectives and paradigm.

% \section{Epistomological Pluralism}
% \textcite{vilner_fundamental_2007} argues that providing “different
% ways of approaching and solving problems” should be a goal of
% introductory programming courses.


% \section{Future Work}

% There continues to be a disconnect between learning sciences and how
% programming paradigms and languages are selected for introductory
% courses. While there is quite a bit of exploration around connections
% between abstract thinking, threshold concepts, etc. and ``success'',
% “success” is not consistently defined, sometimes meaning the ability
% to write programs (e.g. with specifications given), sometimes focusing
% more on the ability to solve problems.

% \subsection{Continued Ties to Learning Sciences}

% The thread of learning science influence split into two branches, one
% further exploring the pattern-based approach and one exploring
% threshold concepts. While developing problem-solving skills remains a
% goal of CS1 only the pattern-based approach thread seems to focus on
% this aspect while the researchers exploring threshold concepts are
% more concerned with overall success in computer science without
% specific mention of problem-solving skills.

% \subsubsection{Problem Solving}

% Several studies pick up the pattern-based focus of programming to
% encourage problem-solving skill development from the
% 1990s\autocite{muller_almost_2004,muller_pattern_2005,muller_patternoriented_2007,caspersen_instructional_2007-1}.

% In one of these studies, the authors draw from three concepts of
% cognitive science: cognitive load theory, cognitive apprenticeship,
% and the theory of worked examples, to design an introductory
% programming course\autocite{caspersen_instructional_2007-1}. It is
% interesting to note that they select these theories based on a human
% cognition model that resulted from computing: separate working and
% long term memory where learning is the process for transforming
% information from working to long-term memory. The course the authors
% describe teachs OOP but they never explain why the chose OOP as the
% paradigm, it is assumed that this is just a given feature of an
% introductory programming course.

% %%TODO: consider focusing on this and dropping threshold concepts and others. Potentally bring in pattern-based pedagogy

% \subsubsection{Threshold Concepts}

% Some studies explore potential threshold
% topics\autocite{meyer_threshold_2005} for computer science and how
% they might be useful for informing course
% design\autocite{mccartney_liminal_2009,boustedt_threshold_2007}. \citeauthor{boustedt_threshold_2007}
% identify object oriented programming and pointers as candidate
% threshold topics based on interviews with students graduating with
% computer science degrees. The two concepts and so these concepts are
% in the context of computer science in general rather than introductory
% programming, thus there is not a clear link between these concepts and
% the goal of problem solving. In addition the authors note that OOP in
% particular is quite broad and suggest further work to narrow the focus
% of that threshold topic to one ore more specific aspects of OOP, such
% as how objects communicate between one another.

% \subsubsection{Objects-first vs. X}
% Much of the literature focuses on the debate between introducing
% object oriented programming in CS1 or whether introductory courses
% should teach a procedural style. Several researchers argue that an
% object-first approach emphasices coding constructs over other goals
% such as problem-solving and design
% \autocite{moritz_objectsfirst_2005,muller_guidelines_2005,reges_back_2006}. \textcite{moritz_objectsfirst_2005}
% propose a “design-first” approach that gets students to analyze
% problems and design solutions before writing any computer code.

% \autocite{vilner_fundamental_2007} found no significant difference in
% student performance on questions relating to recursion and efficiency
% between groups that were given an object-first curriculum vs. groups
% given a procedural-first curriculum. They did observe a significant
% difference between the two groups when asked to design an abstract
% data type: those who were in the objects-first group performed better
% than those in the procedural-first group. However, it is notable that
% the students were only tested on programming knowledge, not their
% ability to analyze problems and design solutions.

% \textcite{reges_back_2006} takes a ``back to basics'' approach eschewing
% the popular OOP and reverts his CS1 course back to a procedural style
% which allows him to put the emphasis back on problem solving.

% \section{Notes for Revisions}

% If teaching OOP concepts is a goal of the course one must ask if
% \emph{implementation} of OOP programs is also a goal.  If writing OOP
% code is a goal and it is unreasonable to assume that students will
% have already mastered syntax and enough understanding of the
% development environment (interpreting compiler messages, debugging,
% etc) then CLT would suggest that extensive \emph{scaffolding} be
% employed to reduce extraneous cognitive load. Being careful to
% sequence simple-to-complex concepts will help reduce \emph{intrinsic}
% cognitive load: i.e. teach bare objects and make sure students have
% began to automate tasks associated with implementing bare objects
% before introducing inheritance.

% Keep in mind that providing scaffolding for syntax and grammar,
% e.g. in the form of worked examples, or implementation in which
% students need to copy/paste implementation code into object template
% code may help with teaching OOP concepts, but would not be aiding
% learning of actually implementing OOP code from scratch. Simply seeing
% the implemented code is likely not enough to aid most students in
% learning this skill themselves and if that is a goal of the course, or
% a subsequent course it is important to focus on that while scaffolding
% other aspects (e.g. providing an object interface and description of
% how the object is intended to function, ask students to write
% implementation). The main challenge with implementing such an
% instructional strategy is that it will likely require coordination
% across courses and instructors as teaching both programming, OOP,
% debugging strategies all at once will likely not be met with much
% success. A possible exception might be if one considers a spiral
% curriculum, but the success of this still depends on close
% coordination across courses and instructors.

% Eric Raymond describes three different kinds of complexity in software
% design: essential complexity, optional complexity, and accidental
% complexity. This has interesting parallels to the different types of
% cognitive load in CLT. In particular the notion of \emph{essential
%   complexity}, like \emph{intrinsic} cognitive load is associated with
% the the solution itself: a solution to a complex problem will be
% essentially complex due to the complexity associated with the problem
% itself.

\vspace{4\baselineskip}\vspace{-\parskip} % Creaters proper 4 blank line spacing.

\printbibliography

\end{document}


